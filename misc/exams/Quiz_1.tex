\documentclass[twocolumn]{article}
\usepackage{enumerate}
\begin{document}

\title{Data Analysis and Python Fundamentals Quiz}
\author{}
\date{}
\maketitle

\section*{Part I: Fundamentals of Data Analysis}

\begin{enumerate}

\item Which of the following best describes the principle of \textbf{modularization} in data analysis?
\begin{enumerate}
    \item Breaking down a program into smaller, reusable parts.
    \item Combining multiple data sources into a single dataset.
    \item Visualizing data using modules.
    \item Encrypting data for security purposes.
\end{enumerate}

\item What task is better suited for a \textbf{GPU}?
\begin{enumerate}
    \item Complicated calculation with many parameters. 
    \item Storing data for long-term use.
    \item Optimizing user input for a web application.
    \item Matrix multiplication in machine learning.
\end{enumerate}

\item Where would you store a lot of experimental data which you need to access
once every 5 years. 
\begin{enumerate}
    \item USB Drive 
    \item SSD 
    \item RAM 
    \item Tape Storage 
\end{enumerate}

\item The \textbf{FAIR} principles stand for:
\begin{enumerate}
    \item Findable, Accessible, Interoperable, Reusable.
    \item Fast, Accurate, Intelligent, Reliable.
    \item Flexible, Adaptable, Integrative, Robust.
    \item Free, Available, Interactive, Real-time.
\end{enumerate}

\item In data analysis, \textbf{data cleaning} refers to:
\begin{enumerate}
    \item Organizing data into tables.
    \item Removing or correcting errors in the dataset.
    \item Deleting unused datasets.
    \item Encrypting data for privacy.
\end{enumerate}

\item The term \textbf{reproducibility} in data analysis means:
\begin{enumerate}
    \item Achieving the same results using the same data and methods.
    \item Sharing data publicly.
    \item Visualizing data in multiple formats.
    \item Using proprietary software.
\end{enumerate}

\item \textbf{Modularity} in programming helps in:
\begin{enumerate}
    \item Increasing code redundancy.
    \item Making code less readable.
    \item Reusing code and simplifying debugging.
    \item Slowing down program execution.
\end{enumerate}

\item \textbf{Data interoperability} refers to:
\begin{enumerate}
    \item The ability to use data across different systems.
    \item The speed at which data is processed.
    \item The size of the dataset.
    \item The security level of data storage.
\end{enumerate}

\item The \textbf{open data} movement encourages:
\begin{enumerate}
    \item Keeping data proprietary.
    \item Charging fees for data access.
    \item Sharing data freely for transparency and collaboration.
    \item Encrypting all data.
\end{enumerate}

\item The purpose of \textbf{data visualization} is to:
\begin{enumerate}
    \item Encrypt data for security.
    \item Represent data graphically to identify patterns and insights.
    \item Store data more efficiently.
    \item Increase data redundancy.
\end{enumerate}

\section*{Part II: Python Fundamentals}

\item What is the correct way to create a list in Python?
\begin{enumerate}
    \item \texttt{my\_list = (1, 2, 3)}
    \item \texttt{my\_list = [1, 2, 3]}
    \item \texttt{my\_list = \{1, 2, 3\}}
    \item \texttt{my\_list = <1, 2, 3>}
\end{enumerate}

\item Which of the following is a \textbf{dictionary} in Python?
\begin{enumerate}
    \item \texttt{my\_dict = \{ 'a':1, 'b':2 \}}
    \item \texttt{my\_dict = ['a', 'b', 'c']}
    \item \texttt{my\_dict = (1, 2, 3)}
    \item \texttt{my\_dict = \{1, 2, 3\}}
\end{enumerate}

\item What is the output of the following code?
\begin{verbatim}
print(2 ** 3)
\end{verbatim}
\begin{enumerate}
    \item \texttt{6}
    \item \texttt{8}
    \item \texttt{9}
    \item \texttt{23}
\end{enumerate}

\item Which of the following is a \textbf{tuple} in Python?
\begin{enumerate}
    \item \texttt{my\_tuple = [1, 2, 3]}
    \item \texttt{my\_tuple = (1, 2, 3)}
    \item \texttt{my\_tuple = \{1, 2, 3\}}
    \item \texttt{my\_tuple = <1, 2, 3>}
\end{enumerate}

\item How do you start a \textbf{function definition} in Python?
\begin{enumerate}
    \item \texttt{function my\_func():}
    \item \texttt{def my\_func():}
    \item \texttt{define my\_func():}
    \item \texttt{func my\_func():}
\end{enumerate}

\item Which of the following is the correct way to write an \textbf{if} statement in Python?
\begin{enumerate}
    \item \texttt{if x > 0 then:}
    \item \texttt{if x > 0:}
    \item \texttt{if (x > 0):}
    \item \texttt{if x > 0;}
\end{enumerate}

\item What is the output of the following code?
\begin{verbatim}
print('Hello' + 'World')
\end{verbatim}
\begin{enumerate}
    \item \texttt{Hello World}
    \item \texttt{HelloWorld}
    \item \texttt{Hello+World}
    \item \texttt{Error}
\end{enumerate}

\item Which of the following is not a valid \textbf{Python data type}?
\begin{enumerate}
    \item \texttt{List}
    \item \texttt{Tuple}
    \item \texttt{Integer}
    \item \texttt{Character}
\end{enumerate}

\item How do you insert comments in Python code?
\begin{enumerate}
    \item \texttt{/* This is a comment */}
    \item \texttt{// This is a comment}
    \item \texttt{\# This is a comment}
    \item \texttt{<!-- This is a comment -->}
\end{enumerate}

\item What does \textbf{IDE} stand for? 
\begin{enumerate}
    \item Integrated Development Environment
    \item Interactive Data Exploration
    \item Intelligent Data Extraction
    \item Integrated Data Encryption
\end{enumerate}

\section*{Part III: Advanced Questions}

\item In Python, what is a \textbf{list comprehension} used for?
\begin{enumerate}
    \item Iterating over lists in a compact form.
    \item Compressing lists to save memory.
    \item Sorting lists in place.
    \item Generating random lists.
\end{enumerate}

\item For a dataset with limited statistics which visualization method best
serves the purpose of understanding the data?
\begin{enumerate}
    \item Pie Chart 
    \item Line Graph 
    \item Scatter Plot 
    \item Histogram with error bars
\end{enumerate}

\end{enumerate}

Open Question: 

Shortly describe how would you structure data analysis for one of
your projects (Lab Reports, BSc Thesis, etc.).


\end{document}
